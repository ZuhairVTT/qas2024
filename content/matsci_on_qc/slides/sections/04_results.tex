\section{Results}

\begin{frame}
\frametitle{Binding Energy Results}
\begin{table}
\begin{tabular}{lcc}
\hline
Method & Binding Energy (eV) & Distance (Å) \\
\hline
Classical DFT & -0.385512 & 3.54 \\
AdaptVQE & -0.385508 & 3.54 \\
Vanilla VQE & -2.325986 & 3.54 \\
\hline
\end{tabular}
\end{table}
\end{frame}

\begin{frame}
\frametitle{Analysis}
\begin{itemize}
    \item AdaptVQE shows excellent agreement with classical DFT
    \item Vanilla VQE shows significant deviation
    \item Binding distance remains consistent across methods
    \item Active space size limitations affect accuracy
\end{itemize}
\end{frame}

\begin{frame}
\frametitle{Future Work}
\begin{itemize}
    \item Expand active space to include more orbitals
    \item Implement error mitigation techniques
    \item Improve convergence of AdaptVQE algorithm
    \item Study larger molecular systems
    \item Integration with quantum centric supercomputers
\end{itemize}
\end{frame}

\begin{frame}
\frametitle{Conclusions}
\begin{itemize}
    \item Successfully demonstrated hybrid quantum-classical workflow
    \item AdaptVQE proves more robust than vanilla VQE
    \item Current limitations:
    \begin{itemize}
        \item Active space size
        \item Convergence challenges
        \item Hardware constraints
    \end{itemize}
    \item Promising path for quantum-accelerated materials science
\end{itemize}
\end{frame} 